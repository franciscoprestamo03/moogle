\documentclass{beamer}

\usetheme{Madrid}
\usecolortheme{default}

\title{Moogle: Advanced Document Search}
\author{Francisco Prestamo}
\date{2023}

\begin{document}

\frame{\titlepage}

\begin{frame}
\frametitle{Overview}
\begin{itemize}
\item Moogle can search within a group of .txt documents.
\item It finds the documents most similar to a given query.
\item Suggests alternatives if query words don't appear in the documents.
\item Implements TF-IDF, Cosine Similarity, and Levenshtein Distance.
\end{itemize}
\end{frame}

\begin{frame}
\frametitle{TF-IDF}
\begin{itemize}
\item TF-IDF stands for Term Frequency-Inverse Document Frequency.
\item It measures the importance of a word in a document within a corpus.
\item The importance increases proportionally to the number of times a word appears in the document and is offset by the number of documents in the corpus that contain the word.
\end{itemize}
\end{frame}

\begin{frame}
\frametitle{Cosine Similarity}
\begin{itemize}
\item Cosine similarity is a measure of similarity between two non-zero vectors.
\item In Moogle, it is used to determine the similarity between the query and documents.
\item It considers the cosine of the angle between two vectors, giving a measure that is not affected by the magnitude of the vectors.
\end{itemize}
\end{frame}

\begin{frame}
\frametitle{Levenshtein Distance}
\begin{itemize}
\item Levenshtein Distance, or edit distance, is a string metric for measuring the difference between two sequences.
\item Moogle utilizes this to suggest alternative queries if the original query words don't appear in the documents.
\item This allows Moogle to handle typos and misspellings in the query.
\end{itemize}
\end{frame}

\begin{frame}
\frametitle{Summary}
\begin{itemize}
\item Moogle provides advanced search capabilities over a corpus of .txt documents.
\item It uses TF-IDF, Cosine Similarity, and Levenshtein Distance to provide accurate results and helpful suggestions.
\end{itemize}
\end{frame}

\begin{frame}
\frametitle{Thank You!}
\centering
Questions?
\end{frame}

\end{document}